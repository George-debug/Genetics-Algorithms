\documentclass{article}

% Language setting
% Replace `english' with e.g. `spanish' to change the document language
\usepackage[english]{babel}

% Set page size and margins
% Replace `letterpaper' with`a4paper' for UK/EU standard size
\usepackage[letterpaper,top=2cm,bottom=2cm,left=3cm,right=3cm,marginparwidth=1.75cm]{geometry}

% Useful packages
\usepackage{amsmath}
\usepackage{graphicx}
\usepackage{tikzit}
\usepackage[colorlinks=true, allcolors=blue]{hyperref}
\usepackage{pgfplots}
\usepackage{float}
\usepackage{amsfonts} 
\usepackage{array}
\usepackage{algorithm}
\usepackage{algpseudocode}
%\algrenewcommand\textproc{}

\input{graphs/style}
\pgfplotsset{compat=1.15}
\input{color_scheme}
\newcommand{\fitnessFunction}[1]{
\centering
\pgfplotstableread[col sep = comma]{#1}{\table}
\begin{tikzpicture}
\begin{axis}[
title = max\_improvement of x,
xlabel = {$x$},
ylabel = {$f(x)$},
axis lines=left,
width=15cm,
height=10cm
]
\addplot[black, domain=0:31, forget plot]{x^3-60*x^2+900*x+100};
\addplot[colorData5, only marks] table [x ={x10}, y = {y10}] {\table};
\addplot[colorData3, only marks] table [x ={x12}, y = {y12}] {\table};
\addplot[colorData2, only marks] table [x ={x7}, y = {y7}] {\table};
\addplot[colorData1, only marks] table [x ={x16}, y = {y16}] {\table};
\legend{$10$, $12$, $7$, $16$}
\end{axis}
\end{tikzpicture}
}
\setlength{\parindent}{0pt}


\title{Differences of Hill Climbing strategies (First Improvement and Best Improvement) in finding global maximum of a cubic function}
\author{George Buțco}

\begin{document}
\maketitle

\begin{abstract}
%The most known strategies for Hill Climbing algorithms are First Improvement and Best Improvement.
In finding global optima, Best Improvement gives the best result in comparison to the other one. Testing these strategies on simple functions is a good way of understanding how they work. In this paper I compare them using the cubic function: $x^3-60x^2+900x+100$ on natural numbers between 0 and 31, showing that Best Improvement is the best strategy for solving this problem.
\end{abstract}

\section{Introduction}
"\textbf{Hill climbing} method is an optimization technique that is able to build a search trajectory in the
search space until reaching the local optima. It only accepts the uphill movement which leads it to
easily get stuck in local optima."\cite{hillClimbing}

At every iteration of this algorithm, we can choose to go with the first or the best successor bigger than the candidate solution \cite{fibi}.

This choice has a big impact on performance and speed. The Best Improvement strategy gives the best result, but it takes a long time to reach it. First Improvement gives a mediocre average optima but within a short time frame. \cite{HCSA}

Even the endian\cite{endians} of the binary representation can affect the algorithm. 

\subsection{Definitions}
\begin{itemize}
    \item
        Let $f(x)$, a numeric function as follows:
        \begin{flalign*}
        f\colon \begin{array}{>{\displaystyle}r @{} >{{}}c<{{}} @{} >{\displaystyle}l} 
                  \mathbb{Z} &\rightarrow& \mathbb{Z} \\ 
                  x &\mapsto& f(x)=x^3 - 60x^2 + 900x + 100
                 \end{array}&&
        \end{flalign*}
        
    \item
        Let $b_n(x)$, a bijective function that returns the binary representation of x as a boolean array of size $n$
        
        $x \in \mathbb{N}$, $x \leq 2^n-1$

        \medskip
        Example:
        
        $b_5(18)$ = [1, 0, 0, 1, 0], for Big Endian.
        
        $b_5(18)$ = [0, 1, 0, 0, 1], for Little Endian.
    
    \item
        Let $d(\texttt{b\_array)}$, a bijective function that returns the natural number represented by the binary array given as a parameter
        
        \medskip
        Example:
        
        $d(\texttt{[1, 0, 0, 1, 0])} = 18$, for Big Endian
        
        $d(\texttt{[0, 1, 0, 0, 1])} = 18$, for Little Endian
        
    \item Let first\_improvement($x$) be defined as follows:
        \begin{algorithm}
        \caption{First Improvement}\label{alg:cap}
        \begin{algorithmic}
        \Require $n \in \mathbb{N}^*$, $x \in \mathbb{N}$
        \Ensure $x \leq 2^n-1$
        \Function{first\_improvement}{x}
            \State $\texttt{b\_array} \gets b_5(x)$
            \For{$i \gets 1$ to $n$}                    
                \State $\texttt{candidate} \gets \texttt{b\_array}$
                \State $\texttt{candidate}[i] = \texttt{candidate}[i] \oplus 1$ \Comment{$\oplus$ is xor operator}
                \If{$d(\texttt{candidate})$ is better than $x$}
                    \State \Return d(\texttt{candidate})
                \EndIf
            \EndFor
            \State \Return x
        \EndFunction
        \end{algorithmic}
        \end{algorithm}
    
    \newpage
    \item Let best\_improvement($x$) be defined as follows:
        \begin{algorithm}
        \caption{Best Improvement}\label{alg:cap}
        \begin{algorithmic}
        \Require $n \in \mathbb{N}^*$, $x \in \mathbb{N}$
        \Ensure $x \leq 2^n-1$
        \Function{best\_improvement}{x}
            \State $rv \gets x$
            \State $\texttt{b\_array} \gets b_5(x)$
            \For{$i \gets 1$ to $n$}                    
                \State $\texttt{candidate} \gets \texttt{b\_array}$
                \State $\texttt{candidate}[i] = \texttt{candidate}[i] \oplus 1$ \Comment{$\oplus$ is xor operator}
                \If{$d(\texttt{candidate})$ is better than $rv$}
                    \State $rv = d(\texttt{candidate})$
                \EndIf
            \EndFor
            \State \Return $rv$
        \EndFunction
        \end{algorithmic}
        \end{algorithm}
        
    \item Let max\_improvement($x$), a function that returns the basin of attraction of $x$.
    
        \begin{algorithm}
        \caption{Max Improvement}\label{alg:cap}
        \begin{algorithmic}
        \Require $\texttt{impr\_func} \in \{\texttt{first\_improvement}, \texttt{best\_improvement}\}$
        \Function{max\_improvement}{x}
            \State $\texttt{next} \gets x$
            \While{$\texttt{next} \neq x$}
                \State $x \gets \texttt{next}$
                \State $\texttt{next} \gets \texttt{impr\_func(next)}$
            \EndWhile
            \State \Return $x$
        \EndFunction
        \end{algorithmic}
        \end{algorithm}
        
    \item Let steps\_improvement($x$), a function that return the number of iterations of max\_improvement

\end{itemize}

\newpage
\subsection{Describe Comparison Method}
The strategies will be compared using:
\begin{itemize}
    \item function $f(x)$, where $x \in [0, 31]$
    \item $n = 5$
\end{itemize}

Every strategy of obtaining global maxima using Hill Climbing will be characterised by:
\begin{itemize}
    \item \textbf{maximum} value of steps\_improvement(x)
    \item \textbf{average} value of steps\_improvement(x)
    \item the \textbf{chance} of max\_improvement(x) to be the global optima
\end{itemize}
\textbf{Hypotheses:}
\begin{enumerate}
    \item First Improvement that uses Big Endian has a smaller average value of steps\_improvement(x) compared to the one that uses Little Endian
    \item Best Improvement has a smaller maximum value of steps\_improvement(x) and a bigger chance of getting the global maximum compared to any First Improvement strategy.
\end{enumerate}


\newpage
\section{Basin of Attraction}
\subsection{First Improvement Big Endian}
\begin{figure}[H]
    \tikzfig{graphs/fi_be}
\end{figure}
\begin{figure}[H]
    \fitnessFunction{"data/fi_be.dat"}
    \caption{First Improvement Big Endian - Basin of Attraction}
\end{figure}

\subsection{First Improvement Little Endian}
\begin{figure}[H]
    \tikzfig{graphs/fi_le}
\end{figure}
\begin{figure}[H]
    \fitnessFunction{"data/fi_le.dat"}
    \caption{First Improvement Little Endian - Basin of Attraction}
\end{figure}

\subsection{Best Improvement}
\begin{figure}[H]
    \tikzfig{graphs/bi}
\end{figure}
\begin{figure}[H]
    \fitnessFunction{"data/bi.dat"}
    \caption{Best Improvement - Basin of Attraction}
\end{figure}

\newpage
\section{Characterisation}
\subsection{First Improvement Big Endian}
\begin{itemize}
    \item \textbf{maximum} value of steps\_improvement(x) is $5$ for $x = 21$
    \item \textbf{average} value of steps\_improvement(x) is $1.75$
    \item the \textbf{chance} of max\_improvement(x) to be the global optima is $43.75\%$ 
\end{itemize}


\subsection{First Improvement Little Endian}
\begin{itemize}
    \item \textbf{maximum} value of steps\_improvement(x) is $5$ for $x = 30$
    \item \textbf{average} value of steps\_improvement(x) is $1.75$
    \item the \textbf{chance} of max\_improvement(x) to be the global optima is $12.15\%$ 
\end{itemize}

\subsection{Best Improvement}
\begin{itemize}
    \item \textbf{maximum} value of steps\_improvement(x) is $5$ for $x = 21$
    \item \textbf{average} value of steps\_improvement(x) is $1.71875$
    \item the \textbf{chance} of max\_improvement(x) to be the global optima is $62.5\%$ 
\end{itemize}

\newpage
\section{Conclusion}

\begin{enumerate}
    \item \textit{"First Improvement that uses Big Endian has a smaller average value of steps\_improvement(x) compared to the one that uses Little Endian"}
    
    \textbf{Conclusion:} The hypothesis is false, at least for this case.
    
    \textbf{Correction:} Both strategies have an average of 1.75. First Improvement Big Endian has a bigger chance of achieving global maxima.
    
    \item \textit{"Best Improvement has a smaller maximum value of steps\_improvement(x) and a bigger chance of getting the global maximum compared to any First Improvement strategy"}
    
    \textbf{Conclusion:} The hypothesis is false, at least for this case.
    
    \textbf{Correction:} Both strategies have the same maximum value of steps\_improvement(x). Best Improvement has a bigger chance of achieving global maxima.
\end{enumerate}

\bigskip

As the result, it was established that Best Improvement does show the best result for each characteristic, which indicates that it is the best possible strategy in this scenario. 


\bibliographystyle{ieeetr}
\bibliography{sample}

\end{document}